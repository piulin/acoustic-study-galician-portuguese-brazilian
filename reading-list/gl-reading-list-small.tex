% !TeX spellcheck = en_US
\documentclass[a4paper,11pt]{article}
\usepackage[left=2.5cm, top=2.5cm, right=2.7cm, bottom=2.4cm]{geometry}
\usepackage[utf8]{inputenc}
\usepackage[english]{babel}
\usepackage{amsfonts}
\usepackage{tipa}
\usepackage{datatool}
\DTLsetseparator{,}

\DTLloaddb{tal}{plo-gl-word-list.csv}
\DTLloaddb{tal2}{fr-gl-word-list.csv}
\usepackage{multicol}

\usepackage{etoolbox}
\usepackage{pgfkeys}
\usepackage{pgfmath}

% code for generating a random permutation
\newcounter{randomListLength}%   current length of our random list
\newcounter{randomListPosition}% current list index
\newcounter{newRandomListElementPosition}% position to insert new element
% insert #1 into the next position of \newRandomList unless the position
% index \randomListPosition is equal to \newRandomListElementPosition in
% which case the \newRandomListElement is added first
\newcommand\randomlyInsertElement[1]{%
    \stepcounter{randomListPosition}%
    \ifnum\value{randomListPosition}=\value{newRandomListElementPosition}%
    \listxadd\newRandomList{\newRandomListElement}%
    \fi%
    \listxadd\newRandomList{#1}%
}
% \randomlyInsertInList{list name}{new list length}{new element}
\newcommand\randomlyInsertInList[3]{%
    \pgfmathparse{random(1,#2)}%
    \setcounter{newRandomListElementPosition}{\pgfmathresult}%
    \ifnum\value{newRandomListElementPosition}=#2\relax%
    \listcsxadd{#1}{#3}%
    \else%
    \def\newRandomList{}% start with an empty list
    \def\newRandomListElement{#3}% and the element that we need to add
    \setcounter{randomListPosition}{0}% starting from position 0
    \xdef\currentList{\csuse{#1}}
    \forlistloop\randomlyInsertElement\currentList%
    \csxdef{#1}{\newRandomList}%
    \fi%
}

% define some pgfkeys to allow key-value arguments
\pgfkeys{/randomList/.is family, /randomList,
    environment/.code = {\global\letcs\beginRandomListEnvironment{#1}
        \global\letcs\endRandomListEnvironment{end#1}
    },
    enumerate/.style = {environment=enumerate},
    itemize/.style = {environment=itemize},
    description/.style = {environment=description},
    seed/.code = {\pgfmathsetseed{#1}}
}
\pgfkeys{/randomList, enumerate}% enumerate is the default

% finally, the code to construct the randomly permuted list
\makeatletter
\newcounter{randomListCounter}% for constructing \randomListItem@<k>'s

% \useRandomItem{k} prints item number k
\newcommand\useRandomItem[1]{\csname randomListItem@#1\endcsname}

% \setRandomItem{k} saves item number k for future use
% and builds a random permutation at the same time
\def\setRandomItem#1\par{\stepcounter{randomListCounter}%
    \expandafter\protected@xdef\csname randomListItem@\therandomListCounter\endcsname{\noexpand\item#1}%
    \randomlyInsertInList{randomlyOrderedList}{\therandomListCounter}{\therandomListCounter}%
}%
\let\realitem=\item
\makeatother
\newenvironment{randomList}[1][]{% optional argument -> pgfkeys
    \pgfkeys{/randomList, #1}% process optional arguments
    \setcounter{randomListLength}{0}% initialise length of random list
    \def\randomlyOrderedList{}% initialise the random list of items
    % Nthing is printed in the main environment. Instead, \item is
    % used to slurp the "contents" of the item into randomListItem@<counter>
    \let\item\setRandomItem%      
}
{% now construct the list environment by looping over the randomly ordered list
    \let\item\realitem
    \setcounter{randomListCounter}{0}
    \beginRandomListEnvironment\relax
    \forlistloop\useRandomItem\randomlyOrderedList
    \endRandomListEnvironment
}

% test compatibility with enumitem
\usepackage{enumitem}
\newlist{Testlist}{enumerate}{1} %
\setlist[Testlist]{label*=\alph*.}
%\setlist{nosep}\parindent=0pt% for more compact output


\begin{document}
	
	\title{Galician reading list.}

	
	\maketitle
	
        \begin{multicols}{3}
\begin{randomList}
    
%    
%    
%    \DTLforeach*{tal}{\name=Plosives,\points=o} {
%    \item Diga \MakeLowercase{\points}\ também.
%    
%}




        
    
    \DTLforeach*{tal}{\name=Plosives,\points=o} {
    \item Diga \MakeLowercase{\points}\ de novo.
    
}
        \DTLforeach*{tal}{\name=Plosives,\points=a} {
        \item Diga \MakeLowercase{\points}\ de novo.
        
    }
    \DTLforeach*{tal}{\name=Plosives,\points=e} {
    \item Diga \MakeLowercase{\points}\ de novo.
    
}

    \DTLforeach*{tal2}{\name=Fricatives,\points=o} {
    \item Diga \MakeLowercase{\points}\ de novo.
    
}
\DTLforeach*{tal2}{\name=Fricatives,\points=a} {
    \item Diga \MakeLowercase{\points}\ de novo.
    
}
\DTLforeach*{tal2}{\name=Fricatives,\points=e} {
    \item Diga \MakeLowercase{\points}\ de novo.
    
}
   
    \end{randomList}
     \end{multicols}
    \newpage
     \begin{multicols}{3}
    \begin{randomList}
       
        
        
        \DTLforeach*{tal}{\name=Plosives,\points=o} {
            \item Diga \MakeLowercase{\points}s\ de novo.
            
        }
        \DTLforeach*{tal}{\name=Plosives,\points=a} {
            \item Diga \MakeLowercase{\points}s\ de novo.
            
        }
        \DTLforeach*{tal}{\name=Plosives,\points=e} {
            \item Diga \MakeLowercase{\points}s\ de novo.
            
        }
        
        \DTLforeach*{tal2}{\name=Fricatives,\points=o} {
            \item Diga \MakeLowercase{\points}s\ de novo.
            
        }
        \DTLforeach*{tal2}{\name=Fricatives,\points=a} {
            \item Diga \MakeLowercase{\points}s\ de novo.
            
        }
        \DTLforeach*{tal2}{\name=Fricatives,\points=e} {
            \item Diga \MakeLowercase{\points}s\ de novo.
            
        }

\end{randomList}
    \end{multicols}

\newpage
\begin{itemize}
    \item O tamén antigo primeiro-ministro do Perú estaba afastado da vida pública hai varios anos devido ó delicado estado de saúde.
    \item As súas formas podem variar de hervas anuais, bianuais ou perenes, sendo raramente arbustos ou árbores.
    
    \item A influencia gravitacional de Neptuno posúe un importante papel no movimento orbital dos obxectos que se encontram além da súa órbita, no Cinturón de Kuiper, poboado por grandes corpos xeados.
\end{itemize}

\end{document}

